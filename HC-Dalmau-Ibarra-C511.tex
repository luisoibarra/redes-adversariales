% This file was converted to LaTeX by Writer2LaTeX ver. 1.6.1
% see http://writer2latex.sourceforge.net for more info
\documentclass[a4paper]{article}
\usepackage[utf8]{inputenc}
\usepackage{amsmath}
\usepackage{amssymb,amsfonts,textcomp}
\usepackage[T1]{fontenc}
\usepackage[english,spanish]{babel}
\usepackage{color}
\usepackage{array}
\usepackage{hhline}
% Outline numbering
\setcounter{secnumdepth}{0}
% Page layout (geometry)
\setlength\voffset{-1in}
\setlength\hoffset{-1in}
\setlength\topmargin{2cm}
\setlength\oddsidemargin{2cm}
\setlength\textheight{25.7cm}
\setlength\textwidth{17.001cm}
\setlength\footskip{0.0cm}
\setlength\headheight{0cm}
\setlength\headsep{0cm}
% Footnote rule
\setlength{\skip\footins}{0.119cm}
\renewcommand\footnoterule{\vspace*{-0.018cm}\setlength\leftskip{0pt}\setlength\rightskip{0pt plus 1fil}\noindent\textcolor{black}{\rule{0.25\columnwidth}{0.018cm}}\vspace*{0.101cm}}
% Pages styles
\makeatletter
\newcommand\ps@Standard{
  \renewcommand\@oddhead{}
  \renewcommand\@evenhead{}
  \renewcommand\@oddfoot{}
  \renewcommand\@evenfoot{}
  \renewcommand\thepage{\arabic{page}}
}
\makeatother
\pagestyle{Standard}
% List styles
\newcommand\liststyleLi{%
\renewcommand\labelitemi{•}
\renewcommand\labelitemii{•}
\renewcommand\labelitemiii{•}
\renewcommand\labelitemiv{•}
}

%
\usepackage{graphicx}
\usepackage[hidelinks,breaklinks=true,backref=page]{hyperref}

\usepackage[spanish]{babel}
\usepackage[utf8]{inputenc}


\begin{document}

\title{26.El desarrollo de las redes adversariales}
\author{Luis Ernesto Ibarra Vázquez C511\and
Luis Enrique Dalmau Coopat C511}
%
%\authorrunning{F. Author et al.}
% First names are abbreviated in the running head.
% If there are more than two authors, 'et al.' is used.
%

\maketitle
\newpage
\tableofcontents
\newpage
\begin{abstract}
	IA, Redes Neuronales, Redes Generativas Adversariales, RGA, GAN, AI, Neural Networks, Generative Adversarial Networks.
\end{abstract}

`` Al enfrentar las redes neuronales entre sí, Ian Goodfellow ha creado una poderosa herramienta de IA. Ahora él y el resto de
nosotros, debemos afrontar las consecuencias.``

\begin{flushright}

\href{https://www.technologyreview.com/author/martin-giles/}{Martin Giles}

\end{flushright}

Pero, ¿de esta frase que debemos conocer?. ¿Qué son las redes neuronales y quién es Ian Goodfellow?

Las redes generativas adversariales o redes generativas antagónicas son un caso particular de las redes neuronales nacidas en el siglo XX.

\section{Antecedentes de las GAN}
Evolución de las Redes Neuronales en Ciencias de la Computación
Vamos a revisar las siguientes redes/arquitecturas:
\begin{itemize}

    \item{1958 – Perceptron}
    \item{1965 – Perceptron Multicapas}
    \item{1980’s
    	\begin{itemize}
    	\item{Neuronas Sigmoidales}
        \item{Redes Feedforward}
        \item{Backpropagation}
        \end{itemize}
        }
    \item{1989 – Convolutional neural networks (CNN) / Recurent neural networks (RNN)}
    \item{1997 – Long short term memory (LSTM)}
    \item{2006 – Deep Belief Networks (DBN): Nace el deep learning}
    \begin{itemize}
    
        \item{Restricted Boltzmann Machine}
        \item{Encoder / Decoder = Auto-encoder}
        
    \end{itemize}
    \item{2014 – Generative Adversarial Networks (GAN)}

\end{itemize}[citar]


\subsection{Perceptron}
Las redes neuronales nacieron con el Perceptron a mediados del siglo XX. El primer tipo de red neuronal y la menos compleja pero un gigantezco primer paso en la dirección correcta.

Entre las décadas de 1950 y 1960 el científico Frank Rosenblatt, inspirado en el trabajo de Warren McCulloch y Walter Pitts creó el Perceptron, la unidad desde donde nacería y se potenciarían las redes neuronales artificiales.

Frank Rossenblatt (11 de julio de 1928-11 de julio de 1971) fue un
psicólogo estadounidense notable en el campo de inteligencia
artificial.En el Laboratorio Cornell Aeronáutico en Búfalo (Nueva
York), donde fue sucesivamente psicólogo investigador, psicólogo
sénior y jefe de la sección de sistemas cognitivos,
también dirigió sus primeros trabajos 
sobre perceptrones, que culminaron en el desarrollo
 y construcción del hardware del 
 perceptron Mark I en
1960.
Este era esencialmente el primer ordenador que podría aprender habilidades nuevas a prueba y error, utilizando un tipo de red neuronal que simula el proceso de pensamiento humano. 

Un perceptron toma varias entradas binarias x1, x2, etc y produce una sóla salida binaria. Para calcular la salida, Rosenblatt introduce el concepto de ``pesos`` w1, w2, etc, un número real que expresa la importancia de la respectiva entrada con la salida. La salida de la neurona será 1 o 0 si la suma de la multiplicación de pesos por entradas es mayor o menor a un determinado umbral.

Sus principales usos son decisiones binarias sencillas, o para crear funciones lógicas como OR, AND. Sin embargo no podia resolver problemas no lineales.


\subsection{Perceptron Multicapas}

En 1969, Minsky y Papert, demuestran que el perceptrón simple y ADALINE no puede resolver problemas no lineales (por ejemplo, XOR). La combinación de varios perceptrones simples podría resolver ciertos problemas no lineales pero no existía un mecanismo automático para adaptar los pesos de la capa oculta. Rumelhart y otros autores, en 1986, presentan la "Regla Delta Generalizada" para adaptar los pesos propagando los errores hacia atrás, es decir, propagar los errores hacia las capas ocultas inferiores. De esta forma se consigue trabajar con múltiples capas y con funciones de activación no lineales. Se demuestra que el perceptrón multicapa es un aproximador universal. Un perceptrón multicapa puede aproximar relaciones no lineales entre los datos de entrada y salida. Esta red se ha convertido en una de las arquitecturas más utilizadas en el momento.

\subsection{Avances en los 80}

\subsubsection{Neuronas Sigmoides}

Para poder lograr que las redes de neuronas aprendieran solas fue necesario introducir un nuevo tipo de neuronas. Las llamadas Neuronas Sigmoides son similares al perceptron, pero permiten que las entradas, en vez de ser ceros o unos, puedan tener valores reales como 0,5 ó 0,377 ó lo que sea. También aparecen las neuronas ``bias`` que siempre suman 1 en las diversas capas para resolver ciertas situaciones. Ahora las salidas en vez de ser 0 ó 1, será d(w . x + b) donde d será la función sigmoide definida como $d(z) = \dfrac{1}{( 1 +e^{-z})}$ . Esta sería la primera función de activación.

La función logística o sigmoidal fue desarrollada como un modelo de crecimiento de población y nombrada ``logística`` por Pierre François Verhulst en las décadas de 1830 y 1840, bajo la dirección de Adolphe Quetelet.
En su primer artículo (1838), Verhulst no especificó cómo ajustaba las curvas a los datos. En su artículo más detallado (1845), Verhulst determinó los tres parámetros del modelo haciendo que la curva pasara por tres puntos observados, lo que produjo una predicción deficiente.
Posterior a este primer descubrimiento, muchos otros autores
 redescubren esta misma funcion en otros ámbitos, llamándolas de
  diversas maneras. En química como modelo de autocatálisis (Wilhelm Ostwald, 1883) y redescubierto como modelo de crecimiento
   demográfico en 1920 por Raymond Pearl y Lowell Reed, publicado como Pearl \& Reed (1920). El término ``logística`` fue revivido por Udny Yule en 1925 y ha sido seguido desde entonces.

[Insertar Imagen]

Con esta nueva fórmula, se puede lograr que pequeñas alteraciones en valores de los pesos (deltas) produzcan pequeñas alteraciones en la salida. Por lo tanto, podemos ir ajustando muy de a poco los pesos de las conexiones e ir obteniendo las salidas deseadas.
[\url{https://en.wikipedia.org/wiki/Logistic_regression#History}]

\subsubsection{Redes Feedforward}

Se les llama así a las redes en que las salidas de una capa son utilizadas como entradas en la próxima capa. Esto quiere decir que no hay loops “hacia atrás”. Siempre se “alimenta” de valores hacia adelante. Hay redes que veremos más adelante en las que sí que existen esos loops (Recurrent Neural Networks).

Además existe el concepto de ``fully connected Feedforward Networks`` y se refiere a que todas las neuronas de entrada, están conectadas con todas las neuronas de la siguiente capa.

\subsubsection{Backpropagation}

Gracias al algoritmo de backpropagation se hizo posible entrenar redes neuronales de multiples capas de manera supervisada. Al calcular el error obtenido en la salida e ir propagando hacia las capas anteriores se van haciendo ajustes pequeños (minimizando costo) en cada iteración para lograr que la red aprenda consiguiendo que la red pueda -por ejemplo- clasificar las entradas correctamente.

El termino retropropagacion (backpropagation) y su uso general en redes neuronales
 se anuncio en Rumelhart, Hinton
  \& Williams (1986), luego se elaboro y popularizo en Rumelhart, 
Hinton \& Williams (1986), pero la tecnica se redescubrio de forma independiente muchas veces y tuvo muchos predecesores que datan a la decada de 1960.

Los conceptos básicos de la retropropagación continua se derivaron en el contexto de la teoría del control por Henry J. Kelley en 1960 y por Arthur E. Bryson en 1961. Utilizaron principios de programación dinámica. En 1962, Stuart Dreyfus publicó una derivación más simple basada únicamente en la regla de la cadena. Bryson y Ho lo describieron como un método de optimización de sistemas dinámicos de varias etapas en 1969. La retropropagación fue derivada por varios investigadores a principios de los años 60 e implementada para ejecutarse en computadoras ya en 1970 por Seppo Linnainmaa. Paul Werbos fue el primero en los EE. UU. en proponer que podría usarse para redes neuronales después de analizarlo en profundidad en su disertación de 1974. Si bien no se aplicó a las redes neuronales, en 1970 Linnainmaa publicó el método general para la diferenciación automática (AD). Aunque es muy controvertido, algunos científicos creen que este fue en realidad el primer paso hacia el desarrollo de un algoritmo de propagación hacia atrás. En 1973, Dreyfus adapta los parámetros de los controladores en proporción a los gradientes de error. En 1974 Werbos mencionó la posibilidad de aplicar este principio a redes neuronales artificiales, y en 1982 aplicó el método AD de Linnainmaa a funciones no lineales.

Posteriormente, el método Werbos fue redescubierto y descrito en 1985 por Parker, y en 1986 por Rumelhart, Hinton y Williams. Rumelhart, Hinton y Williams demostraron experimentalmente que este método puede generar representaciones internas útiles de datos entrantes en capas ocultas de redes neuronales. Yann LeCun propuso la forma moderna del algoritmo de aprendizaje de retropropagación para redes neuronales en su tesis doctoral en 1987. En 1993, Eric Wan ganó un concurso internacional de reconocimiento de patrones a través de retropropagación.

\subsection{Convolutional Neural Network}
Las Convolutional Neural Networks son redes multilayered que toman su inspiración del cortex visual de los animales. Esta arquitectura es útil en varias aplicaciones, principalmente procesamiento de imágenes. La primera CNN fue creada por Yann LeCun y estaba enfocada en el reconocimiento de letras manuscritas.

Yann André LeCun (nacido el 8 de julio de 1960) es un informático franco-estadounidense que trabaja principalmente en los campos del aprendizaje automático , visión por computadora, robótica móvil y neurociencia computacional. Es profesor de plata del Courant Institute of Mathematical Sciences de la Universidad de Nueva York y vicepresidente, científico jefe de IA en Meta ( Anteriormente Facebook ).

Es conocido por su trabajo en reconocimiento óptico de caracteres y visión por computadora utilizando redes neuronales convolucionales (CNN), y es el padre fundador de redes convolucionales. También es uno de los principales creadores de la tecnología de compresión de imagen DjVu (junto con Léon Bottou y Patrick Haffner). Co-desarrolló el lenguaje de programación Lush con Léon Bottou.

Es co-receptor del Premio Turing en 2018 de la ACM junto con Geoffrey Hinton y Yoshua Bengio por su trabajo en aprendizaje profundo.

LeCun, junto con Hinton y Bengio, son conocidos por algunos como los "Padrinos de la IA" y "Padrinos del Aprendizaje Profundo". 

La arquitectura de la primera CNN constaba de varias capas que implementaban la extracción de características y luego clasificar. La imagen se divide en campos receptivos que alimentan una capa convolutional que extrae features de la imagen de entrada (Por ejemplo, detectar lineas verticales, vértices, etc). El siguiente paso es pooling que reduce la dimensionalidad de las features extraídas manteniendo la información más importante. Luego se hace una nueva convolución y otro pooling que alimenta una red feedforward multicapa. La salida final de la red es un grupo de nodos que clasifican el resultado, por ejemplo un nodo para cada número del 0 al 9 (es decir, 10 nodos, se ``activan`` de a uno).

Esta arquitectura usando capas profundas y la clasificación de salida abrieron un mundo nuevo de posibilidades en las redes neuronales. Las CNN se usan también en reconocimiento de video y tareas de Procesamiento del Lenguaje natural. 

\subsection{Long Short Term Memory / Recurrent Neural Network}
Las Long short term memory son un tipo de Recurrent neural network. Esta arquitectura permite conexiones “hacia atrás” entre las capas. Esto las hace buenas para procesar datos de tipo “time series” (datos históricos). En 1997 se crearon las LSTM que consisten en unas celdas de memoria que permiten a la red recordar valores por períodos cortos o largos.

Una celda de memoria contiene compuertas que administran como la información fluye dentro o fuera. La puerta de entrada controla cuando puede entran nueva información en la memoria. La puerta de “olvido” controla cuanto tiempo existe y se retiene esa información. La puerta de salida controla cuando la información en la celda es usada como salida de la celda. La celda contiene pesos que controlan cada compuerta. El algoritmo de entrenamiento -conocido como backpropagation-through-time optimiza estos pesos basado en el error de resultado.

Las LSTM se han aplicado en reconocimiento de voz, de escritura, text-to-speech y otras tareas.


\subsection{Deep Belief Networks (DBN)}
Antes de las DBN en 2006 los modelos con “profundidad” (decenas o cientos de capas) eran considerados demasiado difíciles de entrenar (incluso con backpropagation) y el uso de las redes neuronales artificiales quedó estancado. Con la creación de una DBN que logro obtener un mejor resultado en el MNIST, se devolvió el entusiasmo en poder lograr el aprendizaje profundo en redes neuronales. Hoy en día las DBN no se utilizan demasiado, pero fueron un gran hito en la historia en el desarrollo del deep learning y permitieron seguir la exploración para mejorar las redes existentes CNN, LSTM, etc.

La red de creencias profundas (DBN) es un tipo de red neuronal profunda, que se compone de capas apiladas de máquinas de Boltzmann restringidas (RBM). Es un modelo generativo y fue propuesto por Geoffrey Hinton en 2006.

Geoffrey Everest Hinton (nacido el 6 de diciembre de 1947) es un psicólogo cognitivo e informático británico-canadiense, más conocido por su trabajo en redes neuronales artificiales. Desde 2013, divide su tiempo trabajando para Google (Google Brain) y la Universidad de Toronto. En 2017, cofundó y se convirtió en el principal asesor científico del Vector Institute en Toronto.

Con David Rumelhart y Ronald J. Williams, Hinton fue coautor de un artículo muy citado publicado en 1986 que popularizó el algoritmo de retropropagación para entrenar redes neuronales multicapas, aunque no fueron los primeros en proponer el enfoque. Hinton es visto como una figura destacada en la comunidad de aprendizaje profundo. El espectacular hito del reconocimiento de imágenes de AlexNet diseñado en colaboración con sus alumnos Alex Krizhevsky e Ilya Sutskever para el desafío ImageNet 2012 fue un gran avance en el campo de la visión por computadora.
Hinton recibió el Premio Turing 2018, junto con Yoshua Bengio y Yann LeCun, por su trabajo en aprendizaje profundo y han seguido dando charlas públicas juntos.

Las Deep Belief Networks, demostraron que utilizar pesos aleatorios al inicializar las redes son una mala idea: por ejemplo al utilizar Backpropagation con Descenso por gradiente muchas veces se caía en mínimos locales, sin lograr optimizar los pesos. Mejor será utilizar una asignación de pesos inteligente mediante un preentrenamiento de las capas de la red -en inglés “pretrain”-. Se basa en el uso de la utilización de Restricted Boltzmann Machines y Autoencoders para pre-entrenar la red de manera no supervisada. Ojo! luego de pre-entrenar y asignar esos pesos iniciales, deberemos entrenar la red por de forma habitual, supervisada (por ejemplo con backpropagation).

Se cree que ese preentrenamiento es una de las causas de la gran mejora en las redes neuronales y permitir el deep learning: pues para asignar los valores se evalúa capa a capa, de a una, y no “sufre” de cierto sesgo que causa el backpropagation, al entrenar a todas las capas en simultáneo.

\section{Generatives Adversarial Networks}
INSERTAR IMAGEN DE COMO ESTAN CONFORMADAS LAS REDES GAN

\subsection{Algoritmos Generativos vs. Discriminatorios}

Para entender las GAN, hay que saber cómo funcionan los algoritmos generativos y, para ello, es útil contrastarlos con los llamados algoritmos discriminatorios. Los algoritmos discriminatorios tratan de clasificar los datos de entrada. Es decir, dadas unas características de unos datos, predicen una etiqueta o categoría a la que pertenecen esos datos.

Por ejemplo, dadas todas las palabras de un correo electrónico un algoritmo discriminatorio podría predecir si el mensaje es `spam` o `no spam`. Spam es una de las etiquetas, y la `bolsa de palabras` recopiladas del correo electrónico son las características que constituyen los datos de entrada.

Así que los algoritmos discriminatorios asignan características a las etiquetas. Se refieren únicamente a esa correlación. Una forma de pensar en los algoritmos generativos es que hacen precisamente lo contrario; en lugar de predecir una etiqueta con ciertas características, intentan predecir características con una etiqueta determinada.

Por eso la pregunta que un algoritmo generativo intenta responder es:

``¿Asumiendo que este correo electrónico es spam, ¿Cómo de probable es que sean estas características?``

\subsection{Nacimiento de las GAN}
Una noche de 2014, Ian Goodfellow fue a beber para celebrar con un estudiante de doctorado que acababa de graduarse. A
Les 3 Brasseurs (Los tres cerveceros), un abrevadero favorito de Montreal, algunos amigos le pidieron ayuda con un proyecto bastante desafiante en el que estaban trabajando: un programa que podía crear fotos por sí mismo.

Los investigadores ya estaban usando redes neuronales(HABLAR ANTEs de REDES NEURONALES) como
modelos “generativos” para crear nuevos datos plausibles propios. Pero los resultados a menudo no eran muy buenos: las imágenes de un rostro generado por computadora tendía a ser borroso o a tener errores como orejas faltantes. El plan que tenían los amigos de Goodfellow era utilizar un análisis estadístico complejo de los elementos que componen una fotografía para ayudar a las máquinas a crear
con imágenes por sí mismos. Esto habría requerido una gran cantidad de cálculos numéricos, y Goodfellow les dijo que
simplemente no iba a funcionar.

Pero mientras reflexionaba sobre el problema con su cerveza, se le ocurrió una idea. ¿Qué pasa si enfrentas dos redes neuronales una contra la otra? Sus amigos estaban escépticos, así que una vez que llegó a casa, donde su novia ya estaba profundamente dormida, decidió
darle una oportunidad. Goodfellow codificó en las primeras horas y luego probó su software y este funcionó a la primera.

Lo que inventó esa noche ahora se llama GAN, o "red antagónica generativa". La técnica ha despertado gran
entusiasmo en el campo del aprendizaje automático y convirtió a su creador en una celebridad de la IA.

En los últimos años, los investigadores de IA han logrado un progreso impresionante utilizando una técnica llamada aprendizaje profundo. Suministre a un
sistema de aprendizaje profundo con suficientes imágenes y aprende a, por ejemplo, reconocer a un peatón que está a punto de cruzar una calle. Este enfoque ha hecho posibles cosas como los autos sin conductor y la tecnología conversacional que impulsa a Alexa, Siri y otros asistentes virtuales.

Pero si bien las IA de aprendizaje profundo pueden aprender a reconocer cosas, no han sido buenas para crearlas. El objetivo de las GAN es dar a las máquinas algo parecido a la imaginación.

En el futuro, las computadoras serán mucho mejores para darse un festín con los datos sin procesar que hay por todo el mundo, la ya famosa Big Data, y descubrir lo que necesitan aprender de ellos.

Hacerlo no solo les permitiría hacer dibujos bonitos o componer música; los haría menos dependientes de los humanos
para instruirlos sobre el mundo y la forma en que funciona. Hoy en día, los programadores de IA a menudo necesitan decirle a una máquina exactamente qué hay en los datos de entrenamiento de los cuales se están alimentando: cuáles de ese millón de imágenes contienen un peatón cruzando una calle y cuáles
no. Esto no solo es costoso y requiere mucha mano de obra sino que también limita qué tan bien el sistema maneja incluso las desviaciones leves de lo que ya fue entrenado. En el futuro, las computadoras serán mucho mejores para deleitarse con datos sin procesar y resolver qué necesitan aprender de ellos sin que se les diga.

 Un coche autónomo podría enseñarse a si mismo sobre muchas condiciones diferentes de la carretera sin salir del garaje. Un robot podría anticipar los obstáculos que podría encuentrar en un almacén concurrido sin necesidad de que lo lleven.

Eso marcará un gran salto adelante en lo que se conoce en IA como ``aprendizaje no supervisado``.

Nuestra capacidad de imaginar y reflexionar sobre muchos escenarios diferentes es parte de lo que nos hace humanos. Y cuando los futuros historiadores de la tecnología miren hacia atrás, es probable que vean las GAN como un gran paso hacia la creación de máquinas con una conciencia similar a la humana. Yann LeCun, científico jefe de IA de Facebook, ha llamado a las GAN “la mejor idea en aprendizaje profundo en los últimos 20 años.” 

Otro grande de la IA, Andrew Ng, ex científico jefe de Baidu de China, dice que los GAN representan "un
avance significativo y fundamental” que inspiró a una creciente comunidad global de investigadores.

\subsection{Ian Goodfellow, el padre de las GAN}

Goodfellow ahora es científico investigador en el equipo de Google Brain, en la sede de la compañía en Mountain View,
California. Todavía está sorprendido por su estatus de superestrella, calificándolo de "un poco surrealista". Quizás lo menos sorprendente es que, después de haber hecho su descubrimiento, ahora pasa gran parte de su tiempo tra5bajando contra aquellos que desean usarlo para fines malvados.

La magia de las GAN radica en la rivalidad entre las dos redes neuronales. Imita el tira y afloja entre un falsificador de imágenes y un detective de arte que repetidamente intentan burlarse entre ellos. Ambas redes están entrenadas en el mismo conjunto de datos. El primero, conocido como generador, se encarga de producir resultados artificiales, como fotos o escritura a mano, que sean lo más realistas posible. El segundo, conocido como el discriminador, los compara con imágenes genuinas del conjunto de datos original e intenta determinar cuáles son reales y cuáles son falsos. Sobre la base de esos resultados, el generador ajusta sus parámetros para crear nuevas imágenes. Y así sigue, hasta que el discriminador ya no puede decir lo que es genuino y lo que es falso.
En un ejemplo ampliamente publicitado el año pasado, los investigadores de Nvidia, una empresa de chips que invirtió mucho en IA, entrenaron una GAN para generar imágenes de celebridades imaginarias mediante el estudio de las reales. No todas las estrellas falsas que produjo fueron perfectas, pero algunas fueron impresionantemente realistas. A diferencia de otros enfoques de aprendizaje automático que requieren decenas de miles de imágenes de entrenamiento, las GAN pueden volverse competentes con unos pocos cientos.

Este poder de la imaginación es todavía limitado. Una vez que ha sido entrenado en muchas fotos de perros, un GAN puede generar un
imagen falsa convincente de un perro que tiene, digamos, un patrón diferente de manchas; pero no puede concebir un completamente nuevo
animal. La calidad de los datos de entrenamiento originales también tiene una gran influencia en los resultados. En un ejemplo revelador, un
GAN comenzó a producir imágenes de gatos con letras aleatorias integradas en las imágenes. Porque los datos de entrenamiento
contenía memes de gatos de Internet, la máquina se había enseñado a sí misma que las palabras eran parte de lo que significaba ser un gato.

Las GAN también son temperamentales, dice Pedro Domingos, investigador de aprendizaje automático de la Universidad de Washington. Si el
discriminador es demasiado fácil de engañar, la salida del generador no se verá realista. Y calibrando los dos duelos neurales
Las redes pueden ser difíciles, lo que explica por qué las GAN a veces escupen cosas extrañas, como animales con dos cabezas.

Aún así, los desafíos no han disuadido a los investigadores. Desde que Goodfellow y algunos otros publicaron el primer estudio sobre su
descubrimiento, en 2014, se han escrito cientos de artículos relacionados con GAN. Un fanático de la tecnología incluso ha creado una web
página llamada "Zoológico GAN", dedicada a realizar un seguimiento de las diversas versiones de la técnica que se han
desarrollado.

Las aplicaciones inmediatas más obvias se encuentran en áreas que involucran muchas imágenes, como los videojuegos y la moda:
¿Cómo, por ejemplo, se vería un personaje de un juego corriendo bajo la lluvia? Pero de cara al futuro, Goodfellow cree que las GAN impulsarán avances más significativos. “Hay muchas áreas de la ciencia y la ingeniería en las que debemos
optimizar algo”, dice, citando ejemplos como medicamentos que necesitan ser más efectivos o baterías que deben
ser más eficiente. “Esa va a ser la próxima gran ola”.

En la física de alta energía, los científicos usan poderosas computadoras para simular las probables interacciones de cientos de partículas subatómicas en máquinas como el Gran Colisionador de Hadrones en el CERN en Suiza. Estas simulaciones son lentas y requieren poder de cómputo masivo. Investigadores de la Universidad de Yale y el Laboratorio Nacional Lawrence Berkeley han desarrollado una GAN que, después de entrenarse con los datos de simulación existentes, aprende a generar predicciones bastante precisas de cómo se comportará una partícula en particular, y lo hace mucho más rápido.

[Insertar imagenes raras]

``Hacer que GANS funcione bien puede ser complicado. Si hay fallas, los resultados pueden ser extraños.``

\begin{flushright}
Alec Radford
\end{flushright}


[Insertar imagen de interiores]
``La creación de Goodfellow se puede utilizar para imaginar todo tipo de cosas, incluidos nuevos diseños de interiores.``



{\ttfamily
La investigación médica es otro campo prometedor. Las preocupaciones de privacidad significan que los investigadores a veces no pueden obtener suficientes datos reales de pacientes para, por ejemplo, analizar por qué un medicamento no funcionó. Las GAN pueden ayudar a resolver este problema al generar registros falsos que son casi tan buenos como los reales, dice Casey Greene de la Universidad de Pensilvania. Estos datos podrían compartirse más ampliamente, ayudando a avanzar en la investigación, mientras que los registros reales están estrictamente protegidos. }

Sin embargo, hay un lado más oscuro. Una máquina diseñada para crear falsificaciones realistas es un arma perfecta para los proveedores de noticias falsas que quieren influir en todo, desde los precios de las acciones hasta las elecciones. Las herramientas de IA ya se están utilizando para poner imágenes de rostros de otras personas en los cuerpos de estrellas porno y poner palabras en la boca de los políticos. Las GAN no crearon este problema, pero lo empeorarán.
 
 Hany Farid, que estudia ciencia forense digital en Dartmouth ç College, está trabajando en mejores formas de detectar videos  falsos, como detectar cambios leves en el color de los rostros  causados por la inhalación y la exhalación que las GAN encuentran difíciles de imitar con precisión. Pero advierte que las GAN se  adaptarán a su vez. ``Estamos fundamentalmente en una posición  débil``, dice Farid.


Este juego del gato y el ratón también se desarrollará en la ciberseguridad. Los investigadores ya están destacando el riesgo de los ataques de "caja negra", en los que las GAN se utilizan para descubrir los modelos de aprendizaje automático con los que muchos programas de seguridad detectan el malware. Habiendo adivinado cómo funciona el algoritmo de un defensor, un atacante puede evadirlo e insertar código malicioso. El mismo enfoque también podría usarse para esquivar los filtros de spam y otras defensas.

“Hay muchas áreas de la ciencia y la ingeniería en las que necesitamos optimizar algo. Esa va a ser la próxima gran ola”.

Goodfellow es muy consciente de los peligros. Ahora, al frente de un equipo en Google que se enfoca en hacer que el aprendizaje automático sea más seguro, advierte que la comunidad de IA debe aprender la lección de las oleadas de innovación anteriores, en las que los tecnólogos trataron la seguridad y la privacidad como una idea de último momento. Cuando se dieron cuenta de los riesgos, los malos tenían una ventaja significativa. "Claramente, ya estamos más allá del comienzo", dice, "pero con suerte podemos lograr avances significativos en seguridad antes de que nos adentremos demasiado".

Sin embargo, no cree que haya una solución puramente tecnológica para la falsificación. En su lugar, cree, tendremos que depender de las sociales, como enseñar a los niños el pensamiento crítico al hacer que tomen cosas como clases de oratoria y debate. “En oratoria y debate, estás compitiendo contra otro estudiante”, dice, “y estás pensando en cómo elaborar afirmaciones engañosas o cómo elaborar afirmaciones correctas que sean muy persuasivas”. Puede que tenga razón, pero su conclusión de que la tecnología no puede curar el problema de las noticias falsas no es algo que muchos quieran escuchar.


\section{Referencias}

{\ttfamily
\href{https://es.wikipedia.org/wiki/Red_generativa_antagónica}{https://es.wikipedia.org/wiki/Red\_generativa\_antag\%C3\%B3nica}}

{\ttfamily
\url{https://www.aprendemachinelearning.com/breve-historia-de-las-redes-neuronales-artificiales/}}

{\ttfamily
\url{https://rapidminer.com/glossary/generative-adversarial-networks/}}

{\ttfamily
\url{https://www.technologyreview.com/2018/02/21/145289/the-ganfather-the-man-whos-given-machines-the-gift-of-imagination/}}

\url{https://puentesdigitales.com/2019/04/05/todo-lo-que-necesitas-saber-sobre-las-gan-redes-generativas-antagonicas/}

\url{https://www.aprendemachinelearning.com/breve-historia-de-las-redes-neuronales-artificiales/}

\url{https://es.wikipedia.org/wiki/Perceptr\%C3\%B3n}

\url{https://es.wikipedia.org/wiki/Perceptr\%C3\%B3n_multicapa}

\url{https://es.wikipedia.org/wiki/Frank_Rosenblatt}

\url{https://es.wikipedia.org/wiki/Yann_LeCun}

\url{https://en.wikipedia.org/wiki/Geoffrey_Hinton}
\end{document}
